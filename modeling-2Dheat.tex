% ***********************************************************************************
% Pure LaTeX part to be inserted in a document (be careful of depencies of packages & commands)
% Prepared by Qingan Zhao and Ruitong Zhu under the supervision of Arnaud de La Fortelle
% Fall 2017
% 2D heat diffusion subsection of the modeling part
% ***********************************************************************************

\subgroup{2}{Qingan Zhao and Ruitong Zhu}

\paragraph{Description}
The aim of this part is to describe and model a PDE that describes temperature dynamics in a two-dimensional body via heat conduction.
Basically, heat conduction is the exchange of heat from regions of higher temperatures into regions with lower temperatures, which varies in the transfer rate for different materials.

Consider a thin flat body with a constant thickness $h$ and uniform density $\rho'$. Assume that the faces of the thin body are in perfect insulation, which means there is no heat flow travel in the out-of-plane direction of the body. Hence, heat can only flow in the direction within the plane of the body, which turns into a two-dimensional problem. Then a two-dimensional coordinate system is established such that each point of the body can be described with a coordinate $(x,y)$. Then the (2D-uniform) density of the body is $\rho = \rho' h$. Denote the temperature function of each point by $T$ so that the temperature of the body at position $(x,y)$ and time $t$ are described as $T(x,y,t)$, as shown in Figure~\ref{heatSystem.fig}. The goal is to derive $T(x,y,t)$ when there is no internal heat source.
\begin{figure}[htb]
	\centering
	\includegraphics[width=10cm]{heatSystem.pdf}       
	\caption{System description in 2 dimensions}\label{heatSystem.fig}
\end{figure}

\paragraph{Model}
Consider a small rectangular element of the body with vertices $(x,y)$, $(x+\ud x,y)$, $(x, y+\ud y)$, and $(x+\ud x, y+\ud y)$. The heat flows are shown in Figure~\ref{heatElement.fig}.
\begin{figure}[htb]
	\centering
	\includegraphics[width=8cm]{HeatElement.pdf}       
	\caption{Heat flows in a small rectangular element of the body}\label{heatElement.fig}
\end{figure}

The heat amount $Q$ (i.e the thermal energy) of the rectangular element at time $t$ is: 
\begin{equation}
Q(x,y,t)=C m T(x,y,t)
\end{equation}
where $C$ is called \emph{heat capacity}, which is a supposed to be constant (assuming the material is uniform and temperature do not vary too much); $m = \rho A$ is the mass of the rectangular element where $A$ its surface.

The rate of thermal energy change with respect to time is therefore:
\begin{equation}\label{thermalEnergyChange.eq}
\frac{\partial Q}{\partial t} = C\rho \ud x \ud y\frac{\partial T}{\partial t}
\end{equation}

As shown in Figure~\ref{heatElement.fig}, the incoming flow is $F_1 + F_2 + F_3 + F_4$. Denote the heat flux $\vec q$ in horizontal and vertical directions by $q_x$ and $q_y$, then we have:
\begin{eqnarray} 
F_1 &=& q_x(x,y,t)\ud y\label{flow1}\\
F_2 &=& -q_y(x,y+\ud y,t)\ud x\label{flow2}\\
F_3 &=& q_y(x,y,t)\ud x\label{flow3}\\
F_4 &=& -q_x(x+\ud x,y,t)\ud y\label{flow4}
\end{eqnarray}

Now, we know that according to energy conservation, the thermal energy variation of any small element (as in Equation~(\ref{thermalEnergyChange.eq})) is equal to the total incoming heat flow.  By putting the partial flows as in Equations~(\ref{flow1})-(\ref{flow4}), this conservation principle yields:
\begin{equation}\label{thermalEnergyChange.eq2}
C\rho \ud x \ud y\frac{\partial T}{\partial t} = \ud y [q_x(x,y,t)-q_x(x+\ud x,y,t)]+\ud xh[q_y(x,y,t)-q_y(x,y+\ud y,t)]
\end{equation}

Now, another physical principle, \emph{Fourier's Law}, states that the heat flow is (negatively) proportional to the gradient of temperature:
\begin{equation}\label{FourierLaw.eq}
\vec q = -k\nabla T
\end{equation}
where $k$ is known as the thermal conductivity of the material (also considered as a constant). Then $q_x$ and $q_y$ are expressed as:
\begin{equation}
\begin{split}
q_x=-k\frac{\partial T}{\partial x}\\
q_y=-k\frac{\partial T}{\partial y}
\end{split}
\end{equation}

Hence, Equation~(\ref{thermalEnergyChange.eq2}) can be written as:
\begin{equation}\label{thermalEnergyChange.eq3}
\frac{\partial Q}{\partial t}=k\ud yh\left(\frac{\partial T(x+\ud x,y,t)}{\partial x}-\frac{\partial T(x,y,t)}{\partial x}\right)+k\ud xh\left(\frac{\partial T(x,y+\ud y,t)}{\partial y}-\frac{\partial T(x,y,t)}{\partial y}\right)
\end{equation}

Combine Equation~(\ref{thermalEnergyChange.eq}) and~(\ref{thermalEnergyChange.eq3}):
\begin{equation}
\frac{\partial T(x,y,t)}{\partial t}=\frac{k}{c\rho}\left(\frac{\partial ^2 T(x,y,t)}{\partial x^2}+\frac{\partial ^2 T(x,y,t)}{\partial y^2}\right)
\end{equation}

Denote $k/c\rho$ by $a^2$, and the two-dimensional heat equation can be drawn:
\begin{equation}
\frac{\partial T}{\partial t}=a^2\left(\frac{\partial ^2 T}{\partial x^2}+\frac{\partial ^2 T}{\partial y^2}\right)
\end{equation}

If we would like to solve the PDE in practice, the initial conditions and (or) boundary conditions need to be specified. For initial conditions, assume that in domain $\Omega$:
\begin{equation}
T(x,y,0)=\phi(x,y) 
\end{equation}

As for boundary conditions, we should know either the value or the gradient of the function at some boundaries (i.e., fixed temperature or fixed flow).

For fixed temperature, for example, at the boundary $x=0$, a boundary condition could be stated as follow:
\begin{equation}
T(0, y, t)=0
\end{equation}  
such a boundary condition means the temperature will always be $0$ at the line $x=0$ of the 2D plane.

The boundary condition could aslo be a fixed flow. For example, at the boundary $x=0$:
\begin{equation}
\nabla (0, y, t)=0
\end{equation}
such a boundary condition means that there is no heat transfer at the line $x=0$ of the 2D plane.

When the conditions are given, it is the time to solve the PDE. Most simple PDEs can be solved using analytical methods, such as separation of variables and transform techniques. If the analytical methods do not work, numerical methods can be implemented to find the numerical approximations to the solutions of the PDE. A practical example with boundary conditions will be given in Section 2.3, where the analytical solution will be derived.

Once a system has been solved, we could enhance it by implementing the optimization. One thing we could do is to add some controls. For example, we would like the temperature of a plane less as stable as possible. A proper cost function could be a mathematical expression giving $\ud T/\ud t$ as a function of the external heat sourse. Tools such as Kalman filter can be implemented to do the optimization. A real world example is in aircraft temperature monitoring, for which temperature changes that above a threshold should be warned. 

